\documentclass[conference,a4paper]{IEEEtran}

%opening
\title{Computational intelligence and asthma}

\author{\IEEEauthorblockN{Pablo Vicente-Munuera}
	\IEEEauthorblockA{Bioinformatics MSc\\
		Email: pablovm1990@gmail.com}
}
\usepackage{amsmath}
\usepackage{amsfonts}
\usepackage{amssymb}

\begin{document}

\maketitle

\begin{abstract}

The aim of this work is apply some algorithms typical used in statistics and machine learning to a given dataset. With this, we'll try to find if some of these algorithms can classify our dataset and try to figure something out. The dataset selected for this purpose is an asthma dataset.

\end{abstract}

\section{Introduction}

Asthma is a chronic (long-lasting) inflammatory disease of the airways. In those susceptible to asthma, this inflammation causes the airways to spasm and swell periodically so that the airways narrow. The individual then must wheeze or gasp for air. Obstruction to air flow either resolves spontaneously or responds to a wide range of treatments, but continuing inflammation makes the airways hyper-responsive to stimuli such as cold air, exercise, dust mites, pollutants in the air, and even stress and anxiety.

A few keys from AAAAI (American Academy of Allergy Asthma and Immunology)\cite{AAAAI} who collects all of this information:

\begin{enumerate}
	\item From 2001 through 2009 asthma rates rose the most among black children, almost a 50\% increase, in U.S.A. \cite{asthmaUSA}
	
	\item More than half (53\%) of people with asthma had an asthma attack in 2008. More children (57\%) than adults (51\%) had an attack. 185 children and 3,262 adults died from asthma in 2007, in U.S.A. \cite{asthmaUSA}

	\item An estimated 300 million people worldwide suffer from asthma, with 250,000 annual deaths attributed to the disease.\cite{asthmaGlobal}
	
	\item About 70\% of asthmatics also have allergies.\cite{asthmaGlobal}
	
	\item It is estimated that the number of people with asthma will grow by more than 100 million by 2025. \cite{asthmaGlobal}
\end{enumerate}

But it's not just a problem of the (illogically) called "first world". It also affects to other undeveloped continents such as the African one \cite{asthmaAfrica}:

\begin{itemize}
	\item In 2010, 49.7 million (13.9\%; 95\% CI 9.6-18.3) among children <15 years, 102.9 million (13.8\%; 95\% CI 6.2-21.4) among people aged <45 years, and 119.3 million (12.8\%; 95\% CI 8.2-17.1) in the total population.
\end{itemize}


\subsection{Dataset}

The dataset chosen is the one used by Voraphani N. (2014)\cite{voraphani2014}, in which the main purpose was to identify differentially expressed genes. These genes belongs to different subjects with different classes:

\begin{itemize}
	\item Control: Subjects with no asthma.
	
	\item MMA: Subjects with mild-moderate asthma.
	
	\item SA: severe asthmatic patients.
\end{itemize}

All of these classes represent an expression array with bronchial epithelial cells of Homo Sapiens.
\\
Because we get an expression file, we had to convert it to an arff file (used by weka \cite{weka}). In order to get this file, we have to do some operations before we could some research about it. The first thing, and the most important thing, you must do when you get an expression file, is the normalization of the data. With R and bioconductor \cite{bionductor} (citation here), this purpose is solved. We do the normalization with the background of the array with R.
We could, also, have done differential expression in order to obtain just a few genes (differentially expressed over the others).
\\
After all of this, we obtain 43377 (without array controls) genes and 108 subjects. With this dataset we'll do all of the classification's problems.

\section{Methods}

As said before, the main purpose is classify well and see which genes could contribute more than others We'll analyze these genes so as to see if any of those important genes to our classification problem, have the same relevancy in biology, and, being more specific, in asthma.

Why machine learning \& statistics and not other methods? Well, machine learning and statistics are so powerful. That could be a double-edged sword, but in this case it is not. Although, machine learning, by now, is widely used in the field of bioinformatics and biology.

\subsection{Supervised classification}

In a classification problem, we have a set of elements divided into classes \cite{machineBioinfo}. Given an element (or instance) of the set, a class is assigned according to some of the element's features and a set of classification rules. In our case, we have three classes (Control, MMA, SA) and 108 instances (subjects with asthma or not). Our subjects are labelled with their own class. So we proceed to divide the dataset in two subsets: the training dataset and the test dataset. The training one will be the input (labelled as well) of the classifier. The classifier learn to classify this training dataset, and the output will be a model. With this model, we run again the classifier, but, at this time, we input the test dataset without labels (i.e. no classes). After all of this procedure, we'll obtain a percentage of how good is our classifier.

In order to reduce the bias with the division of the dataset (in training and test), we'll execute the k-fold-cross-validation \cite{CrossValidation}. In this case, the dataset is partitioned into k folds. Each fold is left out of the design process and used as a testing set. The estimate of the error is the overall proportion of the errors committed on all folds.

We've used several paradigms (classifiers) to see which could be fitted more to our data. In the next subsections, they are going to be explained.

\subsubsection{Naive Bayes}

It is built upon the assumption of conditional independence of the predictive variables given the class. 

\begin{equation}
	c^* = \arg \max_c p(C=c) \prod_{i=1}^{n} p(X_i = x_i|C = C)
	\label{eq:naive}
\end{equation}

Which is the reduced formula of this one:

\begin{equation}
	\gamma(x) = \arg \min_k \sum_{c=1}^{r_0} co (k,c) p(c| x_1,\dots, x_n)
	\label{eq:bayes}
\end{equation}

, in which every variable depends on all other and the complexity of the algorithm is too complicated. Naive Bayes (Equation:\ref{eq:bayes}) gives an approximate result, by reducing the dependencies between each variable, and it comes, also, with a time-relaxed version of the algorithm.

\subsubsection{KNN, K-nearest neighbours (IBk)}

Imagine a classification problem, in which you have to divide in (known) classes, seeing only a surface with points and crosses. A way to do that, is starting in one of them and going through all of them and
classify each one by assigning it to the label most frequently represented among the k nearest samples. And you'll have solved the problem by k-nearest neighbours.

\subsubsection{Decision tree: C4.5 (j48)}

The decision tree is as simple as get a tree in which all the leaf nodes are classes and the inner nodes are decision parameters that will help us to determinate whether is one class or another (in the case, there are just two classes). The particularity of C4.5\cite{c45} is that the decision parameters are obtained via information gained ratio:

\begin{equation}
	I(X_i,C)/H(X_i)
\end{equation}

\subsubsection{Logistic regression}

It is based on the logistic function: $f(z) = 1 \frac{1}{1+e{-z}}$. So the equation used here is:

\begin{equation}
	P(C = 1 | x) = \frac{1}{1 + e^{-(\beta_{0} + \Sigma^{n}_{i=1} \beta_{i} x_{i})}}
	\label{eq:logistic}
\end{equation}

where $x$ represents an instance to be classified, and $\beta_0, \beta_1, \dots , \beta_n$ are the parameters of the model. These parameters should be estimated from the data in order to obtain a concrete model.

\subsubsection{Bayesian Networks, TAN}

TAN is based on the mutual information of each variable (or group of variables) with everyone and trying to maximize this number choosing the right variables. So, the mutual information between two variables is given by:

\begin{equation}
	I(X,Y) = \sum_{i=1}^{r_x}\sum_{j=1}^{r_y} p(x_i, y_j) log \frac{p(x_i,y_j)}{p(x_i)p(y_j)}
	\label{eq:tan}
\end{equation}

it measures the reduction of uncertainty of one variable knowing the other one. So, the algorithm consists in building the tree of the variables or group of variables, order by the information gained, which are maximum at this mutual information, until all of the next combinations of variables would not increase the mutual information gained by the last group.

\subsection{Complementary}

book wittenyfrank.pdf

\subsubsection{Random forest}

RandomForest constructs random forests by bagging ensembles of random trees

\subsubsection{Adaboost}

There are many variants on the idea of boosting. We describe a widely used method called AdaBoost.M1 that is designed specifically for classification. Like bagging, it can be applied to any classification learning algorithm. To simplify matters we assume that the learning algorithm can handle weighted instances, where the weight of an instance is a positive number. (We revisit this assump- tion later.) The presence of instance weights changes the way in which a classi- fier’s error is calculated: it is the sum of the weights of the misclassified instances divided by the total weight of all instances, instead of the fraction of instances that are misclassified.

\subsubsection{Multilayer perceptron}

If the data can be separated perfectly into two groups using a hyperplane, it is said to be linearly separable. It turns out that if the data is linearly separable, there is a very simple algorithm for finding a separating hyperplane. This algorithm is the perceptron. And in this case the multilayer perceptron has a very similar approach to the neural net. It also has hidden layers and has to learn the weights of every inputs perceptron. 

\subsubsection{RBF Network}

Other kernel functions can be used instead to implement different nonlinear mappings. Two that are often suggested are the radial basis function (RBF) kernel and the sigmoid kernel. Which one produces the best results depends on the application, although the differences are rarely large in practice. It is interesting to note that a support vector machine with the RBF kernel is simply a type of neural network called an RBF network (which we describe later)

\subsection{Feature subset selection}

One of the main problems of the machine learning algorithms is the complexity problem. You usually get stacked because the algorithms require time and resources, and, very frequently, a lot of each one. So, one of the solution that could be provided is the FSS (feature subset selection). Mainly, FSS reduces the number of variables (or parameters) of our dataset, and the dimensionality associated. Which transform our dataset into a easy one or easier, at least. There three approaches, and we're going to explain each one, in the few next sections.

\subsubsection{Univariate}



\subsubsection{Multivariate}

\subsubsection{Wrapper}

\section{Results}

%The results themselves. 

In this section we'll show all the results we get from the expression set used. The software used to get all of those results is weka \cite{weka}. And all of the results are compared by the ROC curve \cite{ROC}, which is a trusty measure.

\begin{table}[h]
\caption{*: The X values means no results could be obtained due to computational problems. **The K used here is equal to one.}
\centering
\begin{tabular}{c r r r r}
\hline\hline
 & & \multicolumn{3}{|c}{FSS} \\
Method & No filter & \multicolumn{1}{|c}{text}Univariate & Multivariate & Wrapper\\ [0.2ex]
%heading
\hline
Naive Bayes & 0.436 & 0.626 & 0.623 & 0.731 \\
KNN** & 0.594 & 0.644 & 0.67 & 0.781 \\
Logistic &  X* & 0.731 & 0.7 & 0.793\\
Decision Tree & 0.486 & 0.639 & 0.698 & 0.641\\
Bayesian Net & X* & 0.657 & 0.663 & X* \\ [1ex]
\hline
\end{tabular}
\label{table:basicsResults}
\end{table}

We had these problems mentioned before because the initial dataset had, approximately, 43300 variables. One of the first approaches to solve this, was a first univariate filter in which we selected the information gain ratio with respect to their own class. After this filter, all of these classify problems became more time-relaxed, due to the reduction of the number of variables (43300 to 32).

\begin{table}[h]
	\caption{*: K equals to 4; ** K equals to 1.}
	\centering
	\begin{tabular}{c r r r}
		\hline\hline
		& \multicolumn{2}{c}{Multivariate} & \\
		Method & CFS Best first & CFS Genetic & Wrapper\\ [0.2ex]
		%heading
		\hline
		Naive Bayes & 0.623 & 0.599 & 0.74 \\
		KNN & 0.773* & 0.674* & 0.687** \\
		Logistic & 0.699 & 0.675 & 0.824 \\
		Decision Tree & 0.668 & 0.76 & 0.641 \\
		Bayesian Net & 0.667 & 0.674 & 0.678 \\ [1ex]
		\hline
	\end{tabular}
	\label{table:filteredResults}
\end{table}

\begin{table}[h]
	\caption{Results depending the K value on the KNN classifier}
	\centering
	\begin{tabular}{c r r}
		\hline\hline
		K & Univariate & Multivariate\\ [0.2ex]
		%heading
		\hline
		1 & 0.644 & 0.67\\
		2 & 0.735 & 0.72\\
		3 & 0.73 & 0.745\\
		4 & 0.744 & 0.773\\
		5 & 0.762 & 0.807\\
		8 & 0.77 & 0.771 \\ [1ex]
		\hline
	\end{tabular}
	\label{table:knnResults}
\end{table}

\begin{table}[h]
	\caption{Best results on the complementary methods. *: Search method: best first; $\dagger$: Search method: Genetic.}
	\centering
	\begin{tabular}{c r r r r r r}
		\hline\hline
		& & \multicolumn{2}{c|}{FSS} & \multicolumn{3}{c}{Filtered by gain ratio}\\
		Method& None& Univariate& \multicolumn{1}{c|}{CFS*}& CFS*& CFS$\dagger$ & Wrapper\\ [0.2ex]
		%heading
		\hline
		Adaboost(randomForest) & 0.573 & 0.801 & 0.795 & 0.798& 0.829 & 0.835 \\
		Adaboost (J48) & 0.544 & 0.758 & 0.736 & 0.736 & 0.8 & X \\
		Multilayer perceptron & X & 0.788 & 0.792 & 0.778 & 0.697 & 0.83 \\
		RBF Network & 0.393 & 0.675 & 0.726 & 0.726 & 0.693 & X\\
		Random forest & 0.569 & 0.811 & 0.783 & 0.803 & 0.809 & 0.843 \\ [1ex]
		\hline
	\end{tabular}
	\label{table:complementaryResults}
\end{table}

%Mostrar la clasificación entre los bien clasificados de una y otra clase, porque así se podrá ver, que una clase es más fácil de clasificar que otra.

Another important thing which showed up with the output of the results was the classification between one class and the other ones.

\begin{table}[h]
	\caption{Some examples of the difference between classes. Measured by the ROC curve.}
	\centering
	\begin{tabular}{c r r r r}
		\hline\hline
		& \multicolumn{3}{c|}{Classes} & \\
		Method - Filter & Control & MMA & SA & \multicolumn{1}{|c}{Overall}\\ [0.2ex]
		%heading
		\hline 
		KNN (k=1) - Gain ratio & 0.642 & 0.589 & 0.717 & 0.644 \\
		Bayesian Net - Gain ratio \& CFS & 0.802 & 0.591 & 0.697 & 0.667 \\
		Logistic - Wrapper & 0.82 & 0.755 & 0.829 & 0.793 \\
		Decision Tree - No filter & 0.55 & 0.473 & 0.471 & 0.486\\
		Naive Bayes - CFS Best first & 0.766 & 0.534 & 0.664 & 0.623 \\ [1ex]
		\hline
	\end{tabular}
	\label{table:inClassResults}
\end{table}



\section{Discussion}

%The genes discovered.

%\section{Conclusion} is it necessary?

\begin{thebibliography}{1}
	
	%\bibitem{IEEEhowto:kopka}
	%H.~Kopka and P.~W. Daly, \emph{A Guide to \LaTeX}, 3rd~ed.\hskip 1em plus
	% 0.5em minus 0.4em\relax Harlow, England: Addison-Wesley, 1999.
	\bibitem{voraphani2014} Voraphani N, Gladwin MT, Contreras AU, Kaminski N et al. An airway epithelial iNOS-DUOX2-thyroid peroxidase metabolome drives Th1/Th2 nitrative stress in human severe asthma. Mucosal Immunol 2014 Sep;7(5):1175-85.
	
	\bibitem{asthmaDef} "Asthma." MedlinePlus. January 16, 2009 [cited January 20, 2009]. http://www.nlm.nih.gov/medlineplus/asthma.html .
	
	\bibitem{AAAAI} American Academy of Allergy, Asthma, and Immunology (AAAAI) 555 East Wells Street, Suite 1100, Milwaukee, WI 53202-3823. Telephone: (414) 272-6071. http://www.aaaai.org.
	
	\bibitem{asthmaUSA} Centers for Disease Control and Prevention, Vital Signs, May 2011.
	
	\bibitem{asthmaGlobal} World Health Organization. Global surveillance, prevention and control of chronic respiratory diseases: a comprehensive approach, 2007.
	
	\bibitem{asthmaAfrica} Adeloye, Davies et al. “An Estimate of Asthma Prevalence in Africa: A Systematic Analysis.” Croatian Medical Journal 54.6 (2013): 519–531. PMC. Web. 21 Apr. 2015.
	
	\bibitem{machineBioinfo}Larra\~naga, P., Calvo. B., Santana, R., Bielza, C., Galdiano, J., Inza, I., Lozano, J.A., Arma\~nanzas, R., Santaf\'e, G., P\'rez, A. et al. (2006). Machine Learning in Bioinformatics. Briefings in Bioinformatics, 17(1), 86-112.
	
	\bibitem{CrossValidation} Stone M. Cross-validatory choice and assessment of statistical predictions. Journal of the Royal Statistical Society Series B 1974;36:111–47.
	
	\bibitem{naiveBayes} Minsky M. Steps toward artificial intelligence. Transactionson Institute of Radio Engineers 1961;49:8–30.
	
	\bibitem{c45}Quinlan R. C4.5: Programs for Machine Learning. Morgan Kaufmann, 1993.
	
	\bibitem{weka} Ian H. Witten, Eibe Frank, Len Trigg, Mark Hall Geoffrey Holmes, and Sally Jo Cunningham. Weka: Practical machine learning tools and techniques with java implementations. Department of Computer Science. University of Waikato. New Zealand.
	
	\bibitem{ROC} Green DM, Swets JA. Signal Detection Theory and Psychophysics. Wiley, 1974.
	
	
\end{thebibliography}

\end{document}


